%% BioMed_Central_Tex_Template_v1.06
%%                                      %
%  bmc_article.tex            ver: 1.06 %
%                                       %

%%IMPORTANT: do not delete the first line of this template
%%It must be present to enable the BMC Submission system to
%%recognise this template!!

%%%%%%%%%%%%%%%%%%%%%%%%%%%%%%%%%%%%%%%%%
%%                                     %%
%%  LaTeX template for BioMed Central  %%
%%     journal article submissions     %%
%%                                     %%
%%          <8 June 2012>              %%
%%                                     %%
%%                                     %%
%%%%%%%%%%%%%%%%%%%%%%%%%%%%%%%%%%%%%%%%%


%%%%%%%%%%%%%%%%%%%%%%%%%%%%%%%%%%%%%%%%%%%%%%%%%%%%%%%%%%%%%%%%%%%%%
%%                                                                 %%
%% For instructions on how to fill out this Tex template           %%
%% document please refer to Readme.html and the instructions for   %%
%% authors page on the biomed central website                      %%
%% http://www.biomedcentral.com/info/authors/                      %%
%%                                                                 %%
%% Please do not use \input{...} to include other tex files.       %%
%% Submit your LaTeX manuscript as one .tex document.              %%
%%                                                                 %%
%% All additional figures and files should be attached             %%
%% separately and not embedded in the \TeX\ document itself.       %%
%%                                                                 %%
%% BioMed Central currently use the MikTex distribution of         %%
%% TeX for Windows) of TeX and LaTeX.  This is available from      %%
%% http://www.miktex.org                                           %%
%%                                                                 %%
%%%%%%%%%%%%%%%%%%%%%%%%%%%%%%%%%%%%%%%%%%%%%%%%%%%%%%%%%%%%%%%%%%%%%

%%% additional documentclass options:
%  [doublespacing]
%  [linenumbers]   - put the line numbers on margins

%%% loading packages, author definitions

%\documentclass[twocolumn]{bmcart}% uncomment this for twocolumn layout and comment line below
\documentclass[linenumbers]{bmcart}

%%% Load packages
%\usepackage{amsthm,amsmath}
%\RequirePackage{natbib}
\RequirePackage{natbib}% uncomment this for author-year bibliography
%\RequirePackage{hyperref}
\usepackage[utf8]{inputenc} %unicode support
%\usepackage[applemac]{inputenc} %applemac support if unicode package fails
%\usepackage[latin1]{inputenc} %UNIX support if unicode package fails

%%%%%%%%%%%%%%%%%%%%%%%%%%%%%%%%%%%%%%%%%%%%%%%%%
%%                                             %%
%%  If you wish to display your graphics for   %%
%%  your own use using includegraphic or       %%
%%  includegraphics, then comment out the      %%
%%  following two lines of code.               %%
%%  NB: These line *must* be included when     %%
%%  submitting to BMC.                         %%
%%  All figure files must be submitted as      %%
%%  separate graphics through the BMC          %%
%%  submission process, not included in the    %%
%%  submitted article.                         %%
%%                                             %%
%%%%%%%%%%%%%%%%%%%%%%%%%%%%%%%%%%%%%%%%%%%%%%%%%

\usepackage{todonotes}
\usepackage{graphicx}
\usepackage{amsmath}
\usepackage{url}
\usepackage{subcaption}
\usepackage{booktabs}
%\def\includegraphic{}
%\def\includegraphics{}



%%% Put your definitions there:
\startlocaldefs
\endlocaldefs


%%% Begin ...
\begin{document}
	
	%%% Start of article front matter
	\begin{frontmatter}
		
		\begin{fmbox}
			\dochead{Methodology}
			
			%%%%%%%%%%%%%%%%%%%%%%%%%%%%%%%%%%%%%%%%%%%%%%
			%%                                          %%
			%% Enter the title of your article here     %%
			%%                                          %%
			%%%%%%%%%%%%%%%%%%%%%%%%%%%%%%%%%%%%%%%%%%%%%%
			
			\title{Robust Statistical Stopping Criteria for Automated Screening in Systematic Reviews}
			
			%%%%%%%%%%%%%%%%%%%%%%%%%%%%%%%%%%%%%%%%%%%%%%
			%%                                          %%
			%% Enter the authors here                   %%
			%%                                          %%
			%% Specify information, if available,       %%
			%% in the form:                             %%
			%%   <key>={<id1>,<id2>}                    %%
			%%   <key>=                                 %%
			%% Comment or delete the keys which are     %%
			%% not used. Repeat \author command as much %%
			%% as required.                             %%
			%%                                          %%
			%%%%%%%%%%%%%%%%%%%%%%%%%%%%%%%%%%%%%%%%%%%%%%
			
			\author[
			addressref={aff1,aff2},                   % id's of addresses, e.g. {aff1,aff2}
			corref={aff1},                       % id of corresponding address, if any
			%noteref={n1},                        % id's of article notes, if any
			email={callaghan@mcc-berlin.net}   % email address
			]{\inits{MW}\fnm{Max W} \snm{Callaghan}}
			\author[
			addressref={aff1, aff3},
			email={mueller-hansen@mcc-berlin.net}
			]{\inits{FMH}\fnm{Finn} \snm{M\"{u}ller-Hansen}}
			
			%%%%%%%%%%%%%%%%%%%%%%%%%%%%%%%%%%%%%%%%%%%%%%
			%%                                          %%
			%% Enter the authors' addresses here        %%
			%%                                          %%
			%% Repeat \address commands as much as      %%
			%% required.                                %%
			%%                                          %%
			%%%%%%%%%%%%%%%%%%%%%%%%%%%%%%%%%%%%%%%%%%%%%%
			
			\address[id=aff1]{%                           % unique id
				\orgname{Mercator Research Institute on Global Commons and Climate Change}, % university, etc
				\street{Torgauer Straße},                     %
				\postcode{10829}                                % post or zip code
				\city{Berlin},                              % city
				\cny{Germany}                                    % country
			}
			\address[id=aff2]{%
				\orgname{Priestley International Centre for Climate, University of Leeds, Leeds },
				%\street{Dsternbrooker Weg 20},
				\postcode{LS2 9JT}
				\city{Leeds},
				\cny{United Kingdom}
			}
			\address[id=aff3]{%
				\orgname{Potsdam Institute for Climate Impact Research (PIK), Member of the Leibniz Association},
				\street{P.O. Box 60 12 03},
				\postcode{D-14412}
				\city{Potsdam},
				\cny{Germany}
			}
			
			%%%%%%%%%%%%%%%%%%%%%%%%%%%%%%%%%%%%%%%%%%%%%%
			%%                                          %%
			%% Enter short notes here                   %%
			%%                                          %%
			%% Short notes will be after addresses      %%
			%% on first page.                           %%
			%%                                          %%
			%%%%%%%%%%%%%%%%%%%%%%%%%%%%%%%%%%%%%%%%%%%%%%
			
			\begin{artnotes}
				%\note{Sample of title note}     % note to the article
				%\note[id=n1]{Equal contributor} % note, connected to author
			\end{artnotes}
			
		\end{fmbox}% comment this for two column layout
		
		%%%%%%%%%%%%%%%%%%%%%%%%%%%%%%%%%%%%%%%%%%%%%%
		%%                                          %%
		%% The Abstract begins here                 %%
		%%                                          %%
		%% Please refer to the Instructions for     %%
		%% authors on http://www.biomedcentral.com  %%
		%% and include the section headings         %%
		%% accordingly for your article type.       %%
		%%                                          %%
		%%%%%%%%%%%%%%%%%%%%%%%%%%%%%%%%%%%%%%%%%%%%%%
		
		\begin{abstractbox}
			
			\begin{abstract} % abstract
				%\parttitle{First part title} %if any
				Active learning for systematic review screening promises to reduce the human effort required to identify relevant documents for a systematic review. 
				Machines and humans  work together, with humans providing training data, and the machine optimising the documents that the humans screen. This enables the identification of all relevant documents after viewing only a fraction of the total documents. 
				However, current approaches lack robust stopping criteria, so that reviewers do not know when they have seen all or a certain proportion of relevant documents. This means that such systems are hard to implement in live reviews. 
				This paper introduces a workflow with robust and flexible statistical stopping criteria, which offer real work reductions on the basis of a given confidence level of reaching a given recall.
				The stopping criteria are shown on test datasets to achieve a reliable level of recall, while still providing work reductions of on average 17\%. Other methods proposed previously are shown to provide inconsistent recall and work reductions across datasets.
				
			\end{abstract}
			
			%%%%%%%%%%%%%%%%%%%%%%%%%%%%%%%%%%%%%%%%%%%%%%
			%%                                          %%
			%% The keywords begin here                  %%
			%%                                          %%
			%% Put each keyword in separate \kwd{}.     %%
			%%                                          %%
			%%%%%%%%%%%%%%%%%%%%%%%%%%%%%%%%%%%%%%%%%%%%%%
			
			\begin{keyword}
				\kwd{Systematic Review}
				\kwd{Machine Learning}
				\kwd{Active Learning}
				\kwd{Stopping Criteria}
			\end{keyword}
			
			% MSC classifications codes, if any
			%\begin{keyword}[class=AMS]
			%\kwd[Primary ]{}
			%\kwd{}
			%\kwd[; secondary ]{}
			%\end{keyword}
			
		\end{abstractbox}
		%
		%\end{fmbox}% uncomment this for twcolumn layout
		
	\end{frontmatter}

%%%%%%%%%%%%%%%%%%%%%%%%%%%%%%%%%%%%%%%%%%%%%%
%%                                          %%
%% The Main Body begins here                %%
%%                                          %%
%% Please refer to the instructions for     %%
%% authors on:                              %%
%% http://www.biomedcentral.com/info/authors%%
%% and include the section headings         %%
%% accordingly for your article type.       %%
%%                                          %%
%% See the Results and Discussion section   %%
%% for details on how to create sub-sections%%
%%                                          %%
%% use \cite{...} to cite references        %%
%%  \cite{koon} and                         %%
%%  \cite{oreg,khar,zvai,xjon,schn,pond}    %%
%%  \nocite{smith,marg,hunn,advi,koha,mouse}%%
%%                                          %%
%%%%%%%%%%%%%%%%%%%%%%%%%%%%%%%%%%%%%%%%%%%%%%

%%%%%%%%%%%%%%%%%%%%%%%%% start of article main body
% <put your article body there>

%%%%%%%%%%%%%%%%
%% Background %%
%%
	
	%%%%%%%%%%%%%%%%%%%%%%%%
	%% Introduction
	\section*{Background}
	
	Evidence synthesis technology is a rapidly emerging field that promises to change the practice of evidence synthesis work \cite{Westgate2018}.
	Interventions have been proposed at various points in order to reduce the human effort required to produce systematic reviews and other forms of evidence synthesis.
	A major strand of the literature works on screening: the identification of relevant documents in a set of documents whose relevance is uncertain \cite{OMara-Eves2015}. 
	This is a time consuming and repetitive task, and in a research environment with constrained resources and increasing amounts of literature, this may limit the scope of the evidence synthesis projects undertaken.
	Several papers have developed Active Learning (AL) approaches \cite{miwa2014, Wallace2010a, Wallace2010, Jonnalagadda2013, Przybya2018} to reduce the time required to screen documents. This paper sets out how current approaches are  unsuitable in practice, and outlines and evaluates a small modification that would make AL systems ready for live reviews.
	
	Active learning is an iterative process where documents screened by humans are used to train a machine learning model to predict the relevance of unseen papers \cite{Settles2009}.
	The algorithm chooses which studies will next be screened by humans, often those which are likely to be relevant or about which the model is uncertain, in order to generate more labels to feed back to the machine. 
	By prioritising those studies most likely to be relevant, a human reviewer most often identifies all relevant studies - or a given proportion of relevant studies (described by recall: the number of relevant studies identified divided by the total number of relevant studies) - before having seen all the documents in the corpus. 
	The proportion of documents not yet seen by the human when they reach the given recall threshold is referred to as the work saved. This represents the proportion of documents that they do not have to screen, which they would have had to without machine learning.
	
	Machine learning applications are often evaluated using sets of documents from already completed systematic reviews for which inclusion or exclusion labels already exist. 
	As all human labels are known \textit{a priori}, it is possible to simulate the screening process, recording when a given recall target has been achieved.
	In live review settings, however, recall remains unknown until all documents have been screened. 
	In order for work to really be saved, reviewers have to stop screening while uncertain about recall. 
	This is particularly problematic in systematic reviews because low recall increases the risk of bias \cite{Lefebvre2011}.
	The lack of appropriate stopping criteria has therefore been identified as a research gap \cite{bannach-brown2019, Marshall2019}, although some approaches have been suggested. These have most commonly fallen into the following categories:
	\begin{itemize}
		\item \textbf{Sampling criteria:} Reviewers estimate the number of relevant documents by taking a random sample at the start of the process. They stop when this number, or a given proportion of it, has been reached \cite{Shemilt2014}
		\item \textbf{Heuristics:} Reviewers stop when a given number of irrelevant articles are seen in a row \cite{Jonnalagadda2013, Przybya2018}. 
		\item \textbf{Pragmatic criteria:} Reviewers stop when they run out of time \cite{miwa2014}. 
		\item \textbf{Novel automatic stopping criteria:} Recent papers have proposed more complicated novel systems for automatically deciding when to stop screening \cite{Yu2019, DiNunzio2018, Howard2020}
	\end{itemize}

	
	We review the first three classes of these methods in the following section and discuss their theoretical limitations. They are then tested on several previous systematic review datasets.
	We demonstrate theoretically and with our experimental results, that these three classes of methods can not deliver consistent levels of work savings or recall - particularly across different domains, or datasets with different properties \cite{OMara-Eves2015}. We also discuss the limitations of novel automatic stopping criteria, which have all demonstrated promising results, but do not achieve a given level of recall in a reliable or reportable way.   
	Without the reliable or reportable achievement of a desired level of recall, deployment of AL systems in live reviews remains challenging.
	
	This study proposes a system for estimating the recall based on random sampling of remaining documents. 
	We use a simple statistical method to iteratively test a null hypothesis that the recall achieved is less than a given target recall. If the hypothesis can be rejected, we conclude that the recall target has been achieved with a given confidence level and screening can be stopped.
	This allows AL users to predefine a target in terms of uncertainty and recall, so that they can make transparent, easily communicable statements like ``A recall of more than 95\% was achieved with a confidence of 95\%''.
	

	
	In the remainder of the paper, we first discuss in detail the shortcomings of existing stopping criteria. Then, we introduce our new criterion based on a hypergeometric test. We evaluate our stopping criteria, and compare their performance with heuristic and sampling based criteria on real-world systematic review datasets on which AL systems have previously been tested \cite{Cohen2006, Yu2019, Terasawa2009, Castaldi2009}.
	
	\section*{Methods Review}


	We start by explaining the sampling and heuristic based stopping criteria and discussing their methodological limitations. 
	
	\subsection*{Sampling Based Stopping Criteria}
	
	The stopping criterion suggested by Shemilt et al. \cite{Shemilt2014} involves establishing the Baseline Inclusion Rate (BIR), by taking a random sample at the beginning of screening. 
	The BIR is used to estimate the number of relevant documents in the whole dataset. 
	Reviewers continue to screen until this number, or a proportion of it corresponding to the desired level of recall, is reached.
	
	
	However, the estimation of the BIR fails to correctly take into account sampling uncertainty \footnote{Although Shemilt et al. \cite{Shemilt2014} employ a method  to choose a sample size based on uncertainty, they fail to acknowledge the potential implications for recall of their choice. Their margin of error of 0.0025 and observed proportion of relevant studies of 0.0005 translate to estimates of $400 \pm 451$ relevant results. To reduce the margin of error to $\pm 5\%$ of estimated relevant studies, they would have had to screen 638,323 out of 804,919 results. See the notebook \url{https://github.com/mcallaghan/rapid-screening/blob/master/analysis/bir_theory.ipynb} that accompanies this paper for a detailed discussion}. 
	This uncertainty is crucial, as errors can have severe consequences. If the estimated number of relevant documents is even one unit above the true value, then no work savings will be achieved. If the number of relevant documents is underestimated, then the recall achieved will be less than 100\%.
	
	The number of relevant documents drawn without replacement from a finite sample of documents follows the hypergeometric distribution. 
	Figure \ref{fig:bir_error} shows the distribution of the predicted number of documents after drawing 1,000 documents from a total of 20,000 documents, where 500 documents (2.5\%) are relevant. The left shaded portion of the graph shows all the cases where the recall will be less than 95\%. This occurs 35.91\% of the time. The right shaded portion of the graph shows the cases where the number of relevant documents is overestimated and no work savings could be made. This occurs 46.24\% of the time. In only 17.85\% of cases can work savings be achieved while still achieving a recall of at least 95\%. 
	
	Figure \ref{fig:bir_error_distribution} shows the probability distribution of these errors according to the sample size. Even in very large samples both types of error remain frequent, and the risk of saving no work at all remains close to 50\%.
	This shows how baseline estimation inevitably offers poor reliability, both in terms of recall and in work saved.


	\begin{figure*}
	\centering
	\begin{subfigure}[b]{0.475\textwidth}   
		\centering 
		\includegraphics[width=\textwidth]{2_figs_bir_errors.pdf}
		\caption[]%
		{{\small Error types after 1,000 documents \\}}    
		\label{fig:bir_error}
	\end{subfigure}
	\hfill
	\begin{subfigure}[b]{0.475\textwidth}   
		\centering 
		\includegraphics[width=\textwidth]{2_figs_bir_error_distribution.pdf}
		\caption[]%
		{{\footnotesize Error types across sample sizes}}    
		\label{fig:bir_error_distribution}
	\end{subfigure}

	\caption{\small Distribution of under- or over-estimation errors using the BIR sampling method in a dataset of 20,000 documents of which 500 are relevant. Panel (a) shows the probability distribution of the estimated number of relevant documents after a sample of 1,000 documents. Panel (b) shows the probability of each type of error according to the sample size.} 
	\label{bir-sampling}
\end{figure*}

\begin{figure}
	\includegraphics[width=\linewidth]{2_figs_proportions_1.png}
	\includegraphics[width=\linewidth]{2_figs_proportions_2.png}
	\caption{Similar low proportions of relevant documents in unseen documents with different consequences for recall. The top bar shows a random distribution of relevant documents (green) and irrelevant documents (red) at a given proportion of relevance. The bottom bar  shows distributions of relevant and irrelevant documents in hypothetical sets of seen (right) and unseen (left - transparent) documents.}
	\label{unseen-proportions}
\end{figure}


		
\subsubsection*{Heuristic Stopping Criteria}

Some studies give the example of heuristic stopping criteria based on drawing a given number of irrelevant articles in a row \cite{Jonnalagadda2013, Przybya2018}. 
We take this as a proxy for estimating that the proportion of documents remaining in the unseen documents is low, as the probability of observing 0 relevant documents in a given sample (analogous to a set of consecutive irrelevant results) is inversely related to the number of relevant documents in the population.
We find this a promising intuition, but argue that 1) it ignores uncertainty, as discussed in relation to the previous method; 
2) it lacks a formal definition; and 3) it misunderstands the significance of a low proportion of relevant documents in estimating the recall.

Figure \ref{unseen-proportions} illustrates this third point. 
We show two scenarios with identical low proportions of relevant documents observed in the unseen documents.
In the top figure, machine learning (ML) has performed well, and 74\% of the screened documents were relevant. 
In the bottom figure,  ML has performed less well, and only 26\% of the screened documents were relevant.
In both cases, only 2\% of unseen documents are relevant, but 2\% of a larger number means more relevant documents are missed.
Recall is not simply a function of the relevance of unseen documents, but also of the number of unseen documents. 
This also means that where ML has performed well (as in the top figure), a low proportion of relevant documents in those not yet checked is indicative of lower recall than where ML has performed less well.
Likewise, where the proportion of relevant documents in the whole corpus is low, a similarly low proportion of relevant documents is likely to be observed, even when true recall is low. 
This shows us that even a perfect estimator of the relevance of unseen documents is insufficient on its own to provide sufficient information about when to stop screening.



\subsubsection*{Pragmatic stopping criteria}

Wallace et al. \cite{Wallace2010a} develop a ``simple, operational stopping criterion'': stopping after half the documents have been screened. Although the criterion worked in their experiment, it is unclear how this could be generalised, and its development depended on knowledge of the true relevance values. 
Jonnalagadda and Petitti \cite{Jonnalagadda2013} note that ``the reviewer can elect to end the process of classifying documents at any point, recognizing that stopping before reviewing all documents involves a trade-off of lower recall for reduced workload'', although clearly the reviewer lacks information about probable recall.

\subsubsection*{Novel automatic stopping criteria}
Two examples come from the information retrieval literature. Di Nunzio \cite{DiNunzio2018} presents a novel automatic stopping criterion based on BM25, although recall reported is ``often between 0.92 and 0.94 and consistently over 0.7''.
Yu and Menzies \cite{Yu2019} also present a stopping criterion based on BM25 which allows the user to target a specific level of recall. 
However, reviewers are not given the opportunity to specify a confidence level, and for two of the four datasets in which they tested their criteria, the median achieved recall at a stopping criteria targeting 95\% recall was below 95\%. In each case, the reliability of the estimate is dependent on the performance of the model.

Finally, Howard et al. \cite{Howard2020} present a method to estimate recall based on the the number of irrelevant documents $D$ observed in a list of documents since the $\delta th$ previous relevant document. 
They reason that this should follow the negative binomial distribution based on the proportion of remaining relevant documents $p$, and use this information to estimate $\hat{p}$, and with this, the total number of relevant articles and the estimated recall. 
However, since screening is a form of sampling without replacement, the negative hypergeometric distribution should be preferred to the negative binomial. Further the authors do not give sufficient information to reproduce their results, providing neither code (they describe their own proprietary software), nor an equation for $\hat{p}$. Additionally, the criterion requires a tuning parameter $\delta$, which users may have insufficient information to set optimally. Lastly, their method does not quantify uncertainty, but can only claim that the method ``\textit{tends} to result in a conservative estimate of recall'' (emphasis ours).

These last examples are promising developments, but fail to take into account the needs of live systematic reviews, where the reliability of and ease of communication about recall are paramount, and the results are independent of model performance. In the following, we explain our own method, which provides clearly communicable estimates of recall, which manage uncertainty in a way robust to model performance.

\subsection*{Methods}

\subsection*{A Statistical Stopping Criterion for Active Learning}

	In our screening setup, we start off with $N_T$ documents that are potentially relevant. $\rho_{tot}$ of these documents are actually relevant, but we don't know this value \textit{a priori}. As we screen relevant documents we include them, so $\rho_{seen}$ represents the number of relevant documents screened, and $\tau$ recall is given by 
	
	\begin{equation}
		\tau = \frac{\rho_{seen}}{\rho_{tot}}
	\end{equation}
	
	We set a target recall $\tau_{tar}$ and a confidence level $\alpha$. 
	We want to keep screening until $\tau \geq \tau_{tar}$, and devise a hypothesis test to estimate whether this is the case with a given level of confidence. 
	We do this	based on interrupting the active-learning process and drawing a random sample from the remaining unseen documents. 
	We first describe this test, before showing how a variation on the test can be used to decide when to begin drawing a random sample. 

	\subsubsection*{Random Sampling}
	
	At the start of the sample, $N_{AL}$ is the number of documents seen during the active learning process, and $N_s$ is the number of documents remaining, so that 
	
	\begin{equation}
	N_s = N_T - N_{AL}
	\end{equation} 
	
	We refer to the number  of relevant documents seen during active learning as $\rho_{AL}$, and the number of remaining relevant documents as $K$. We do not know the value of $K$ but know that it is given by the total number of relevant documents minus the number of relevant documents seen during active learning.
	
	\begin{equation}
	K = \rho_{tot} - \rho_{AL}
	\end{equation}
	
	With each draw, $n$ is the number of documents drawn and $k$ is the number of relevant documents drawn. The number of relevant documents seen is updated by adding the number of relevant documents seen since sampling began to the number of relevant documents seen during active learning.
	
	\begin{equation}
	\rho_{seen} = \rho_{ML} + k
	\end{equation}
	
	We proceed to form a null hypothesis that the true value of recall is less than our target recall:
	
	\begin{equation}
	H_0 : \tau < \tau_{tar}
	\end{equation}
	
	Because we are sampling without replacement, we can use the hypergeometric distribution to find out the probability of observing $k$ relevant documents in a sample of $n$ documents from a population of $N$ documents of which $K$ are relevant. We know that $k$ is distributed hypergeometrically:
	
	\begin{equation}
	X \sim Hypergeometric(N, K, n)
	\end{equation}
	
	We introduce a hypothetical value for $K$, which we call $K_0$. This represents the minimum number of relevant documents remaining at the start of sampling compatible with our null hypothesis that recall is below our target.
	
	\begin{equation}
	K_0 = \lfloor \frac{\rho_{seen}}{\tau_{tar}}-\rho_{AL}+1 \rfloor
	\end{equation}
	
	Our null hypothesis can thereby reformulated: the true number of relevant documents in the sample is greater than our hypothetical value.
	
	\begin{equation}
	H_0 : K > K_{tar}
	\end{equation}
	
	We test this by calculating the probability of observing $k$ or fewer relevant from the hypergeometric distribution given by $K_{tar}$, using the cumulative distribution function.
	
	\begin{equation}
	p = P(X \leq k)$, where $X \sim Hypergeometric(N_s,K_{tar},n)
	\end{equation}
	
	Because $P(X \leq k)$, is strictly decreasing for values below $K_{tar}$, this gives the maximum probability of observing $k$ for all values of K compatible with our null hypothesis.
	
	If the maximum probability of obtaining our observed results given our null hypothesis is below our $1-\alpha$, then we can reject our null hypothesis and stop screening. We can report the likelihood that we achieve a recall below our target as being less than $1-\alpha$.
	
	


	\subsubsection*{Nonrandom sampling}
	
	In order to decide when to begin a random sample, we employ nonrandom sampling, where we treat previously screened documents as random samples. The distribution of relevant documents among previously screened documents is clearly not random, as documents predicted to be relevant are prioritised. It is reasonable to assume, though, that the density of relevant documents is greater among previously screened documents than among remaining unseen documents. This would make the following estimates conservative. 
	
	After reviewing each document, $N_{seen}$ documents have been screened, and $N_{unseen}$ documents are yet to be seen. 
	We treat $i = 1 \dots N_{seen}$ of the previously screened documents as separate random samples, and calculate $p$, using the method above, for each sample. 
	We take the minimum across all samples $p_{min}$ as the lowest probability that our null hypothesis is correct. 
	If $p_{min}$ is less than $1-\frac{\alpha}{2}$, we switch to random sampling. 
	This method is an imprecise heuristic used to decide when to start random sampling. 
	However, we also test its performance as a separate stopping criterion.
	We calculate $p_{min}$ for the remaining documents as if we had not switched to random sampling and record the recall and work saved when $p_{min} < 1 - \alpha$. 
	We present these in the results below as the nonrandom sampling criterion.


	\medskip
	
	\begin{figure}
		\includegraphics[width=0.7\linewidth]{2_figs_flow.pdf}
		\caption{A workflow for active learning in screening with a statistical stopping criterion}
		\label{flow}
	\end{figure}

	\subsection*{Evaluation}
	
	We evaluate each of the criteria discussed on real world test data, operationalising the heuristic stopping criteria with 50, 100, and 200 consecutive irrelevant records. We run 100 iterations on each dataset and record the following measures.
	\begin{itemize}
		\item \textbf{Actual Recall}: The recall when the stopping criteria was met
		\item \textbf{WS-SC}: Work saved when the stopping criteria was met
		\item \textbf{Additional Burden}: the work saved when the criterion was triggered subtracted from the work saved when the recall target was actually achieved.

	\end{itemize}
	For simplicity, we use a basic SVM model \cite{Cortes95, Pedregosa2011}, with 1-2 word n-grams taken from the document abstracts used as input data. We start with random samples of 200 documents (we do not employ Shemilt et al's methods for identifying the ``optimal'' sample size, as we showed these in the methods section to be unhelpful). Subsequently, we ``screen'', that is, we reveal the labels of, batches of the 20 documents with the highest predicted relevance scores, retraining the model after each batch. Theoretically, using smaller batch sizes could mean that the recall target is achieved more quickly, but this is a trade-off between computational time spent training, and the speed at which the algorithm can ``learn''. However this is a modelling choice which may affect work saved, but not recall. Each criterion is evaluated after each document is ``screened''.
	 For our criteria, we set the target recall value to 95\% and the confidence level to 95\%.
	
	
	%\subsubsection*{Evaluation Data}
	\begin{table}
		\input{1_tables_datasets.tex}
		\caption{Dataset properties}
		\label{tab:data}
	\end{table}
	
	The systematic review datasets used for testing are described in table \ref{tab:data}. We use the seminal collection of systematic reviews used to develop machine learning applications for document screening by Aaron Cohen and co-authors in 2006 \cite{Cohen2006}, along with the widely used Proton Beam \cite{Terasawa2009} and COPD \cite{Castaldi2009} datasets, and computer science datasets used to test FASTREAD \cite{Yu2019}. Testing on datasets with different properties and from different domains is key to establishing criteria appropriate for general use. Choosing as broad as possible data also prevents us from being able to ``tune'' our machine learning approach in ways that may work well for specific datasets but not generalise well. Work savings, even maximum work savings are therefore below the state of the art recorded for each of these datasets. In this way we can show how well the criteria perform even when the model performs badly.
	
	%As a last step, we explain the factors that increase or decrease the performance of our criteria, using a simple regression model.
	
	All computational steps required to reproduce this analysis are documented online at \url{https://github.com/mcallaghan/rapid-screening}.
	
	\section*{Results}
	
	Figure \ref{recall-wss} shows the actual recall and work savings achieved when each stopping criteria has been satisfied. 
	For comparison, we also include the results that would have been achieved with \textit{a priori} knowledge of the data, that is, the work saved when the 95\% recall target was actually reached. In a live systematic review, reviewers would never know when this had been reached, but these are the work savings most often reported in machine learning for systematic review screening studies.
	

    \begin{figure*}
	\centering
	\begin{subfigure}[b]{0.475\textwidth}   
		\centering 
		\includegraphics[width=\textwidth]{2_figs_jointplot_hyper.pdf}
		\caption[]%
		{{\small Hypergeometric sampling \\}}    
		\label{fig:hyper}
	\end{subfigure}
	\hfill
	\begin{subfigure}[b]{0.475\textwidth}   
		\centering 
		\includegraphics[width=\textwidth]{2_figs_jointplot_nrs.pdf}
		\caption[]%
		{{\footnotesize Pseudorandom hypergeometric sampling}}    
		\label{fig:nrs}
	\end{subfigure}
	\vskip\baselineskip
	\begin{subfigure}[b]{0.475\textwidth}
		\centering
		\includegraphics[width=\textwidth]{2_figs_jointplot_bir.pdf}
		\caption[Network2]%Example-Image
		{{\small Baseline Sampling}}    
		\label{fig:bir}
	\end{subfigure}
	\hfill
	\begin{subfigure}[b]{0.475\textwidth}  
		\centering 
		\includegraphics[width=\textwidth]{2_figs_jointplot_ih_50.pdf}
		\caption[]%
		{{\small 50 consecutive irrelevant results}}    
		\label{fig:ih_50}
	\end{subfigure}
	\vskip\baselineskip
	\begin{subfigure}[b]{0.475\textwidth}   
		\centering 
		\includegraphics[width=\textwidth]{2_figs_jointplot_ih_200.pdf}
		\caption[]%
		{{\small 200 consecutive irrelevant results \\}}    
		\label{fig:ih_200}
	\end{subfigure}
	\hfill
	\begin{subfigure}[b]{0.475\textwidth}   
		\centering 
		\includegraphics[width=\textwidth]{2_figs_jointplot_pf.pdf}
		\caption[]%
		{{\footnotesize \textit{a priori} knowledge}}   
		\label{fig:pf}
	\end{subfigure}

	\caption{\small Distribution of recall and work saved after each stopping criteria. Green dots show results for datasets with less than 1,000 documents, orange dots show datasets with 1,000 - 2,000 documents, and blue dots show datasets with more than 2,000 documents.} 
	\label{recall-wss}
\end{figure*}

\begin{figure*}
	\centering
		\begin{subfigure}[b]{0.475\textwidth}   
		\centering 
		\includegraphics[width=\textwidth]{2_figs_jointplot_burden_hyper.pdf}
		\caption[]%
		{{\small Hypergeometric sampling \\}}    
		\label{fig:hyper_ab}
	\end{subfigure}
	\hfill
	\begin{subfigure}[b]{0.475\textwidth}   
		\centering 
		\includegraphics[width=\textwidth]{2_figs_jointplot_burden_nrs.pdf}
		\caption[]%
		{{\footnotesize Pseudorandom hypergeometric sampling}}    
		\label{fig:nrs_ab}
	\end{subfigure}
	\hfill
	\vskip\baselineskip
	\begin{subfigure}[b]{0.475\textwidth}
		\centering
		\includegraphics[width=\textwidth]{2_figs_jointplot_burden_bir.pdf}
		\caption[Network2]%Example-Image
		{{\small Baseline Sampling}}    
		\label{fig:bir_ab}
	\end{subfigure}
	\hfill
	\begin{subfigure}[b]{0.475\textwidth}  
		\centering 
		\includegraphics[width=\textwidth]{2_figs_jointplot_burden_ih_50.pdf}
		\caption[]%
		{{\small 50 consecutive irrelevant results}}    
		\label{fig:ih_50_ab}
	\end{subfigure}
	\vskip\baselineskip

	\begin{subfigure}[b]{0.475\textwidth}   
		\centering 
		\includegraphics[width=\textwidth]{2_figs_jointplot_burden_ih_100.pdf}
		\caption[]%
		{{\small 100 consecutive irrelevant results }}    
		\label{fig:ih_200_ab}		

	\end{subfigure}
	\hfill
	\begin{subfigure}[b]{0.475\textwidth}   
		\centering 
		\includegraphics[width=\textwidth]{2_figs_jointplot_burden_ih_200.pdf}
		\caption[]%
		{{\footnotesize 200 consecutive irrelevant results }}    
		\label{fig:ih_100_ab}
	\end{subfigure}
	
	\caption{\small Distribution of recall and additional burden after each stopping criterion. Additional burden is the work saved when the criterion was triggered minus the work saved when the target was reached. Coloring of data points as in Fig. \ref{recall-wss}.}
	\label{recall-burden}
\end{figure*}


	
	Both the random sampling and the pseudorandom sampling criteria achieve the target threshold of 95\% in more than 95\% of cases. In fact, the pseudorandom sampling criterion  outperforms the random sampling criterion with respect to both recall and work savings, saving a mean of 17\% of the work compared to 15\%, and missing the target in only 0.95\% compared to 3.29\% of cases. In theory, the pseudorandom sampling criteria is conservative if the assumption holds that documents chosen by machine learning are not less likely to be relevant than those chosen at random. Based on our experiments, this assumption seems reasonable, and accounts for the higher recall. Because the pseudorandom sampling criterion can flexibly choose its sample, whereas the random criterion has to wait for a random sample to be triggered, the criterion is also triggered earlier, as it can make use of more data. This accounts for the higher work savings.
	
	The baseline sampling criteria (Figure \ref{fig:bir}) misses the 95\% recall target in 39.67\% of cases, while the most common work saving is 0\%. This is in line with our expectations that, due to random sampling error, the expected number of documents will often be over-estimated or under-estimated, resulting in zero work savings or poor recall.
	
	The Heuristic stopping criteria, both for 50 consecutive irrelevant results (Figure \ref{fig:ih_50} - IH50), and for 200 irrelevant results (Figure \ref{fig:ih_200}) also perform unreliably. Although the mean work saved for IH50 is 41\%, the target is missed in 39\% of cases. The cases below the horizontal grey line indicate instances where work has been saved at the expense of achieving the recall target.
	
	\begin{figure}
		\includegraphics[width=0.9\linewidth]{2_figs_wss_nrs.pdf}
		\caption{Work saved for the pseudorandom sampling method in each dataset. Labels show the number of relevant documents and the total number of documents. The datasets are presented in order of the number of documents. The whiskers represent the 5th and 95th percentiles. The grey line shows work savings of 5\%. }
		\label{wss}
	\end{figure}
	
 	\begin{figure*}
		\centering
		\begin{subfigure}[b]{0.475\textwidth}
			\centering
			\includegraphics[width=\textwidth]{2_figs_h0_paths_Radjenovic.pdf}
			\caption[Network2]%Example-Image
			{{\small Radjenovic}}    
			\label{fig:Radjenovic}
		\end{subfigure}
		\hfill
		\begin{subfigure}[b]{0.475\textwidth}  
			\centering 
			\includegraphics[width=\textwidth]{2_figs_h0_paths_ProtonBeam.pdf}
			\caption[]%
			{{\small ProtonBeam}}    
			\label{fig:ProtonBeam}
		\end{subfigure}
	\vskip\baselineskip
		\begin{subfigure}[b]{0.475\textwidth}
			\centering
			\includegraphics[width=\textwidth]{2_figs_h0_paths_Statins.pdf}
			\caption[Network2]%Example-Image
			{{\small Statins}}    
			\label{fig:Statins}
		\end{subfigure}
		\hfill
		\begin{subfigure}[b]{0.475\textwidth}  
			\centering 
			\includegraphics[width=\textwidth]{2_figs_h0_paths_Triptans.pdf}
			\caption[]%
			{{\small Triptans}}    
			\label{fig:Triptans}
		\end{subfigure}
	
		\caption{\small The path of recall (yellow) and the p-value of H0 for four different datasets} 
		\label{H0paths}
	\end{figure*}
	
	In figure \ref{recall-burden} we rescale the x axis, calling it additional burden, which is simply the work saved when the criterion is triggered minus the work saved when the recall target was actually achieved. This measure indicates whether the stopping criterion was triggered too early (negative values), or too late (positive values). The figure directly highlights the tradeoffs involved in deciding when to stop screening: For our criteria, there is mostly a small additional burden which comes with the necessity to make sure the desired recall target has been reached and reject the null hypothesis that this has not been the case. For the other criteria, there are many cases in which additional burden is negative, i.e. the criterion has been triggered too early. In these cases, however, the desired recall is hardly ever reached.
	
	To help explain the different work savings that were observed in our experiments, we show the distribution of work savings from our pseudorandom criterion for each dataset in figure \ref{wss}. In general, higher work savings are possible when the total number of documents is larger. However, in datasets with a low proportion of relevant documents, many documents need to be screened to achieve a high confidence that there are only few relevant documents remaining in the unseen ones. Therefore, smaller work savings are possible. 
	
	Figure \ref{H0paths} shows the recall and the probability of the null hypothesis for the best performing iteration of four datasets. Although the 95\% recall target is achieved very quickly in the Radjenovic dataset, the null hypothesis cannot be excluded until much later. This is because the dataset has only 47 relevant documents out of a population of 5,999. After the 95\% recall target was achieved, 45 out of 47 relevant documents had been seen and 5,029 documents remained. The null hypothesis was therefore that 3 or more of these 5,029 documents were relevant, which requires a lot of evidence to disprove. The burden of proof was smaller in the case of the Proton Beam dataset: at the point that the 95\% recall threshold was reached, the null hypothesis to disprove was that a minimum of 13 out of 3,369 remaining documents were relevant. 
	
	The Statins and Triptans datasets show how the criterion performs when the machine learning model has performed poorly in predicting relevant results. In each case, 95\% recall is achieved with close to 20\% of documents remaining. With fewer documents remaining, it takes fewer screening decisions to rule out the possibility that the number of relevant documents left is incompatible with the achievement of the recall target.
	
	\section*{Discussion}
	
	Our results show that it is possible to use machine learning to achieve a given level of recall with a given level of confidence. The tradeoff for achieving recall reliably is that the work saving achieved is less than the maximum possible work saving. However, for large datasets with a significant proportion of relevant documents, the additional effort required to satisfy the criterion will be small compared to the work saved by using machine learning. This makes the approach well suited to broad topics with lots of literature. In other words, it is precisely where machine learning will be most useful that the additional effort will be small.
	
	Different use cases for machine learning enhanced screening may also carry different requirements for recall, or different tolerances for uncertainty. These can be flexibly accommodated within our stopping criterion. Importantly, the ability to make probabilistic statements about the chance of achieving a given recall target makes it possible to clearly communicate the implications of using machine learning enhanced screening to readers and reviewers who are not machine learning specialists. This is extremely important in live systematic reviews. 
	
	Our criteria have the further advantage that they are independent of the choice or performance of the machine learning model. If a model performs badly at discerning relevant from irrelevant results, the only consequence will be that the work saved will be low. With other criteria this may result in poor recall. 
	When using machine learning for screening, poor recall can result in biased results, while low work savings represent no loss to the reviewer as compared to not using machine learning.
	
	So far, systematic review standards have no way of accommodating screening with machine learning. 
	We hope that the reliability and clarity of reporting offered by our stopping criteria make them suitable for incorporation into standards, so that machine learning for systematic review screening can fulfil its promise of reducing workload and making more ambitious reviews tractable.
	
	\section*{Conclusion}
	
	This paper demonstrates the drawbacks of existing stopping criteria for machine learning approaches to document screening, particularly with regard to reliability. We propose a simple method that delivers reliable recall, independent of machine learning approach or model performance. Our robust statistical stopping criteria allow users to easily communicate the implications of their use of machine learning, making machine learning enhanced screening ready for live reviews.
	
	
\begin{backmatter}
	
	\section*{Ethics approval}
	Not applicable.
	
	\section*{Consent for publication}
	Not applicable.
		
	\section*{Availability of data and materials} 
	All computational steps required to reproduce this analysis are documented online at \url{https://github.com/mcallaghan/rapid-screening}.
	
	\section*{Competing interests}
	The authors declare that they have no competing interests.
	
	\section*{Author's contributions}
	MC designed the research and conducted the experiments. FMH contributed to the development of the statistical basis for the stopping criterion. Both authors wrote and edited the manuscript.
	
	\section*{Acknowledgements}
	Max Callaghan is supported by a PhD scholarship from the Heinrich Böll Foundation. Finn M\"{u}ller-Hansen acknowledges funding from the German Federal Ministry of Research and Education within the Strategic Scenario Analysis (START) project (grant reference: 03EK3046B).
		
	\bibliography{3_bib_mendeley}
	\bibliographystyle{vancouver}
	
\end{backmatter}
\end{document}