\documentclass[9pt,aspectratio=169]{beamer}
\mode<presentation>
\usepackage[T1]{fontenc}
\usepackage{color}
\usepackage{graphicx}
\usepackage{natbib}
\usepackage{tikz}
\usetikzlibrary{shapes.geometric}
\usepackage{xmpmulti}
\usepackage{animate}
\usepackage{tcolorbox}
\usepackage{amsmath}
\usepackage{xcolor, soul}
\usepackage{gensymb}
\usepackage{appendixnumberbeamer}

%\definecolor{links}{HTML}{2A1B81}
%\hypersetup{urlcolor=links} % Does not apply color to href's
%\hypersetup{colorlinks,urlcolor=links} 

\usetheme{Boadilla}
%\usecolortheme{seahorse}

\usefonttheme{professionalfonts}

\title[Stopping Criteria]{Statistical stopping criteria for automated screening in systematic reviews}
\date{20 October 2022}
\author{Max Callaghan, \underline{Finn Müller-Hansen}}
\institute[MCC]{
	\includegraphics[height=1cm,width=2cm]{images/MCC_Logo_RZ_rgb.jpg}
}



\newtheorem*{remark}{}


\bibliographystyle{apalike}

\begin{document}
	\begin{frame}
	\titlepage
\end{frame}

%%%%%%%%%%%%%%%%%%%%%%
% Motivation
%\begin{frame}{Motivation}
%
%\end{frame}

%%%%%%%%%%%%%%%%%%%%%%
% RITL
\begin{frame}{ML-assissted document screening}

\begin{columns}
	\begin{column}{0.618\linewidth}
		\begin{itemize}
			\item<1-> Growing number of ``researcher-in-the-loop'' machine learning applications for screening documents for systematic reviews \citep{omara-eves_using_2015, van_de_schoot_open_2021}.
			\item<2-> By using machine learning to prioritise documents likely to be relevant, we can achieve high levels of recall without screening all documents.
			\item<3-> BUT, given we do not know \textit{a priori} the true number of relevant documents, we need criteria when to stop screening.
			\item<4->Stopping too early can lead to huge biases in reviews.
		\end{itemize}
	\end{column}
	\begin{column}{0.382\linewidth}
		\begin{figure}
			\only<2>{\includegraphics[width=\linewidth]{images/idealised_stopping.pdf}}
			\only<3>{\includegraphics[width=\linewidth]{images/potential_stopping.pdf}}
			\only<4>{\includegraphics[width=\linewidth]{images/early_stopping.pdf}}
%			\only<2>{\includegraphics[width=\linewidth]{images/random.pdf}}
%			\only<3>{\includegraphics[width=\linewidth]{images/ml_prio.pdf}}
%			\only<4>{\includegraphics[width=\linewidth]{images/ml_prio.pdf}}
		\end{figure}
	\end{column}
\end{columns}

\end{frame}

%%%%%%%%%%%%%%%%%%%%%%
% Urn
\begin{frame}{A statistical stopping criterion}



\begin{columns}
	
	\begin{column}{0.382\linewidth}
		\begin{figure}
			\only<1-3>{\includegraphics[width=.8\linewidth]{images/Keats_urn.jpg}}
			\only<4->{\includegraphics[width=\linewidth]{../images/h0_paths_ProtonBeam.pdf}}
			
		\end{figure}
	\end{column}
	\begin{column}{0.618\linewidth}
		\begin{itemize}
			\item<1-> Our stopping criterion works by treating documents as if they were white (not relevant) and red (relevant) marbles drawn from an urn without replacement.
			\item<2-> The hypergeometric distribution describes the probability of observing \textbf{k} red marbles in a sample of \textbf{n} marbles, given an urn with \textbf{N} marbles, of which \textbf{K} are red.
			\item<3-> We formulate a null hypothesis $H_0$ that a given recall target (e.g. 95\% of relevant documents) has been \textbf{missed}. 
			\item<4-> We calculate a p-score for $H_0$ and stop screening if this falls below a selected threshold.
		\end{itemize}
	\end{column}

\end{columns}

\medskip

\bigskip

\only<5->{Note: ML-prioritisation means documents are not drawn at random, which makes our test conservative.}

\end{frame}


%%%%%%%%%%%%%%%%%%%%%%
% Results
\begin{frame}{Results}

We run simulations on 20 complete systematic review datasets to test our criterion.

\begin{columns}
	\begin{column}{0.618\linewidth}
		\begin{itemize}
			\item<1-> \textit{Theoretically achievable} work savings in the datasets varies widely (higher for larger datasets - blue dots)
			\item<2-> Existing stopping criteria can result in very low recall (examples: 50 consecutive irrelevant articles, using baseline inclusion rate)
			\item<4-> Our criterion generated work savings with reliably conservative performance wrt our recall target.
		\end{itemize}
	\end{column}
	\begin{column}{0.382\linewidth}
		\begin{figure}
			\only<1>{\includegraphics[width=\linewidth]{../manuscript/2_figs_jointplot_pf.pdf}
				\caption{A priori knowledge}
			}
			\only<2>{\includegraphics[width=\linewidth]{../manuscript/2_figs_jointplot_ih_50.pdf}
				\caption{50 consecutive irrelevant articles}
			}
			\only<3>{\includegraphics[width=\linewidth]{../manuscript/2_figs_jointplot_bir.pdf}
				\caption{Estimating baseline inclusion rate}
			}
			\only<4>{\includegraphics[width=\linewidth]{../manuscript/2_figs_jointplot_nrs.pdf}
				\caption{Our criterion}
			}

		\end{figure}
	\end{column}
\end{columns}


\end{frame}


%%%%%%%%%%%%%%%%%%%%%%
% Conclusion
\begin{frame}{Conclusion}

We provide a stopping criterion that works on any model and can be included in any tool: \url{https://github.com/mcallaghan/rapid-screening/blob/master/analysis/hyper_criteriaR.md}.

\medskip

In practice, we see huge work savings for large datasets!

\medskip

Future work: use noncentral hypergeometric distribution to generate a less conservative test and save more work.

\begin{center}
	\line(1,0){250}
	
	\medskip
	
	\textbf{Thanks!}
	
	\medskip
	
	Contact: \url{callaghan@mcc-berlin.net, mueller-hansen@mcc-berlin.net}
	
	Twitter: \url{https://twitter.com/MaxCallaghan5}
	
\end{center}

\end{frame}

%%%%%%%%%%%%%%%%%%

\appendix


\begin{frame}{References}

\bibliography{rapid-review}

\end{frame}

\begin{frame}{Applications and extensions}

\begin{columns}

\column{0.33\textwidth}
\only<1->{
	\begin{figure}
		\includegraphics[width=\linewidth]{images/doebbeling_ws.png}
	\end{figure}
}

\column{0.33\textwidth}
\only<2->{
	\begin{figure}
		\includegraphics[width=\linewidth]{images/doebbeling_p.png}
	\end{figure}
}


\column{0.33\textwidth}

\small

\begin{itemize}
	\item<1-> We have used the stopping criterion to generate massive savings (77\%) in real projects
	\item<2-> If rejecting our $H_0$ was less labour intensive we could have saved around 82\%
	\item<3-> Using a biased urn could help create a more precise criterion
\end{itemize}

\end{columns}

\end{frame}			

%%%%%%%%%%%%%%%%%%%%%%
% Other criteria
\begin{frame}{Baseline Inclusion Rate}

\begin{columns}
	\begin{column}{0.618\linewidth}
		\begin{itemize}
			\item<1-> If we try to estimate the baseline inclusion rate, we will get it wrong most of the time. Overestimating results in 0 work savings, while underestimating results in less than target recall.
			\item<2-> Wrongness decreases with larger sample sizes, but bad outcomes remain most frequent.
		\end{itemize}
	\end{column}
	\begin{column}{0.382\linewidth}
		\begin{figure}
			\only<1>{\includegraphics[width=\linewidth]{../manuscript/2_figs_bir_errors.pdf}
			}
			\only<2>{\includegraphics[width=\linewidth]{../manuscript/2_figs_bir_error_distribution.pdf}
}
			
		\end{figure}
	\end{column}
\end{columns}

\end{frame}

\begin{frame}{Theory I}
We form a null hypothesis that the target level of recall has not been achieved

\begin{equation}
H_0 : \tau < \tau_{tar}
\end{equation}

To operationalise this, we come up with a hypothetical value of $K$ which is the lowest value compatible with our null hypothesis

\begin{equation}
K_{tar} = \lfloor \frac{\rho_{seen}}{\tau_{tar}}-\rho_{AL}+1 \rfloor
\end{equation}

In other words, if there were $K_{tar}$ or more relevant documents in the urn when sampling began, the $\rho_{al}$ relevant we identified before sampling, and the $k$ we drew from the urn would not be enough to meet our target recall level.

\medskip 

The cumulative distribution function gives us the probability of observing what we observed, if our null hypothesis were true

\begin{equation}
p = P(X \leq k) \text{, where } X \sim Hypergeometric(N,K_{tar},n)
\label{eq:p-value}
\end{equation}

\end{frame}


\end{document}